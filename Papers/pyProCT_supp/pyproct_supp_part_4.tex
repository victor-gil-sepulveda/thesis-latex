\subsection{Results file}

The results file summarizes the whole clustering process and its results.
It contains extra information that can be processed afterwards with
the results viewer in order to gain more insight about the used quality
functions and criteria.


\subsubsection{Best clustering}

This section contains only one property that holds the identification string of the best
clustering.
\begin{lstlisting}[caption={Best clustering result ID},firstnumber=1,language=json]
{
	"best_clustering": String
}
\end{lstlisting}

\begin{description}
\item [best\_clustering] Id of the best clustering.
\end{description}

\subsubsection{Files}

This section stores a list of file objects, containing the details
of all generated files (included the results file itself):

\begin{lstlisting}[caption={The file details section of the results file},firstnumber=1,language=json]
{
	"files": List(FileDetailsObject)
}
\end{lstlisting}
A FileDetails object has the following structure:

\begin{lstlisting}[caption={FileDetails object},firstnumber=1,language=json]
{
	"path": String,
	"type": String in ['text','image'],
	"description": String
}
\end{lstlisting}

\begin{description}
\item [{path}] Complete path of the file
\item [{type}] `text' if the file is a human-readable txt file, or `image'
if it is a viewable image file.
\item [{description}] Contains a very brief description of the file contents.
\end{description}

\subsubsection{Trajectories}

A list containing objects with details of the trajectories used in
this clustering:

\begin{lstlisting}[caption={The trajectory details section of results file},firstnumber=1,language=json]
{
	"trajectories": List(TrajectoryDetailsObject)
}

\end{lstlisting}

Each TrajectoryDetails object has the following structure:

\begin{lstlisting}[caption={TrajectoryDetails object},firstnumber=1,language=json]
{
	"source": String,
	"conformations": Integer,
	"atoms": Integer
}
\end{lstlisting}

\begin{description}
\item [{source}] Complete path of this trajectory.
\item [{conformations}] Number of conformations (`model' sections) of
the pdb.
\item [{atoms}] Number of atoms of each conformation.
\end{description}

\subsubsection{Workspace}

Is a copy of the same section in the parameters file.

\subsubsection{Scores}

Contains the score value of each criterion for all non-filtered clusterings.

\begin{lstlisting}[caption={Possible score section for a clustering process that used two criteria and ended with two candidates},firstnumber=1,language=json] 
"scores": {
	"criterion_0": {
		"clustering_0001": 0.9822,
		"clustering_0002": 0.9935
	},
	"criterion_1": {
		"clustering_0001": 0.9150,
		"clustering_0002": 0.9229
	}
} 
\end{lstlisting}


\subsubsection{Timing}

Compiles the values of all timer objects in a list. These can be used
to rapidly assess the performance of the execution.

A timer object has the following structure:

\begin{lstlisting}[caption={Timer object},firstnumber=1,language=json]
{
	"name": String,
	"elapsed": Real
}
\end{lstlisting}

\begin{description}
	\item [{name}] Name of the step being checked.
	\item [{elapsed}] The duration of the step in seconds.
\end{description}

\subsubsection{Clustering information}

The clustering information section is indeed composed of 2 similar
sections: `selected' stores all non-filtered clusterings, and `not\_selected'
holds the ones that were filtered.

Both sections contain JSON objects with details of the generated clustering.
A clustering object is indexed by its id, and contains the following
subsections:

\textbf{clustering::clusters } List(ClusterObject)

A list of the clusters forming the clustering. See the input file
section \textbf{clustering\allowbreak::generation\allowbreak::clusters }for a description
of the format of a cluster object.

\textbf{clustering::total\_number\_of\_elements} Integer

The amount of clustered elements.

\textbf{clustering::number\_of\_clusters} Integer

The length of the clusters list.

\textbf{evaluation } Evaluation

An evaluation object containing the values of all calculated queries
and quality functions for this clustering.

\textbf{type} String in {[}`dbscan', `gromos', `hierarchical', `kmedoids',
`spectral', `random'{]}

Indicates which algorithm generated this clustering.

\textbf{parameters} ParametersObject

Parameters used by the algorithm to generate this clustering (as detailed
in the Algorithms section).

Clusterings under the `\_selected' key have slightly different
contents. They share all fields but the `evaluation' one, which changes
to `reasons'.

\textbf{reasons } List(ReasonObject)

Holds a list of reasons why the clustering was not considered for
evaluation. each reason object looks like this:

\begin{lstlisting}[caption={Reason object},
firstnumber=1, 
language=json, 
breakatwhitespace=true 
%,literate={\_}{}{0\discretionary{_}{\\}{}}
]
{
    "reason": String in ["TOO_FEW_CLUSTERS",
                        "TOO_MUCH_CLUSTERS", 
                           "TOO_MUCH_NOISE",
                "EQUAL_TO_OTHER_CLUSTERING"],
    "data": ReasonDataObject
}
\end{lstlisting}

\begin{description}
	\item [`reason'] The reason to exclude this clustering from the evaluation
	step. `TOO\_FEW\_CLUSTERS' and `TOO\_MUCH\_CLUSTERS' mean that the
	number of clusters is not into the allowed range. `TOO\_MUCH\_NOISE'
	means that the clustering had too much noise. `EQUAL\_TO\_OTHER\_CLUSTERING'
	means that an exact clustering has been already generated (with other
	algorithm or parameters).
	\item [`data'] Gives details about the reason to eliminate this clustering.
	For the first 3 cases, it will store the maximum/minimum value of the
	range and the current value for that clustering.
\end{description}

\begin{lstlisting}[caption={This clustering was not used because it had fewer clusters \(1\) than the minimum number of clusters allowed \(6\)},firstnumber=1,language=json]
{
	"reason": "TOO_FEW_CLUSTERS",
	"data": {
		"current": 1,
		"minimum": 6
	}
}
\end{lstlisting}

Data objects for `EQUAL\_TO\_OTHER\_CLUSTERING' reasons will only
store the id of the repeated clustering.

\begin{lstlisting}[caption={},firstnumber=1,language=json]
{
	"reason": "EQUAL_TO_OTHER_CLUSTERING",
	"data": {
		"id": "clustering_0003"
	}
}
\end{lstlisting}
