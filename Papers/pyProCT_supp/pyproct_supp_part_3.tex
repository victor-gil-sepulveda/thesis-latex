\subsection{Input script}

pyProCT uses as input a human-readable JSON text file. Its file
extension can be arbitrarily chosen, however pyProCT will generally
produce this kind of files with a `.json' extension.

The input file describes how the software interacts with the underlying
hardware where it is being executed, how the clustering exploration
will be performed, and finally, which kind of post process operations
will be performed after the clustering has been obtained. This section
will be structured in 4 subsections, one for each of the main subsections
of the script (`global', `data', `clustering' and `postprocess').

Unless otherwise specified, each property or object will be represented
by a descriptor. This descriptor contains the label of the property
or JSON object (subsections), information about its data type, and
possible dependencies.

A label consists of a double colon separated list of the names of
the objects that encompass the property or subsection, followed by
its name. For instance, \textbf{x::y::z} corresponds to the JSON object:

\begin{lstlisting}[firstnumber=1,language=json,caption={Simple JSON object.}]
"x":{ 	
	"y":{
		"z":{}
	}
}
\end{lstlisting}

The nature of z will be specified after the label. Possible tags are: 

\begin{description}
	\item [{Subsection}] The item is another JSON object that can contain another
	subsections (JSON objects) and properties. 
	\item [{Integer}] The item is an integer value property. 
	\item [{Real}] The item is a real value property. 
	\item [{String}] The item is a string property. 
	\item [{List}] The item is an array of elements. 
\end{description}

If the value of the property has to be chosen from a limited list
of possible choices, this values will be enumerated in a ``value
in {[}choices{]}\textquotedblright{} clause. ``Optional\textquotedblright{}
will be added to the end of the descriptor if the property is optional.

Finally, if a property or subsection needs it, a dependency clause
will be added in front of the label. This dependency clause contains
the label it depends on and the value it needs to have, or, if it
just depends on its previous definition, an ``is defined\textquotedblright{}
clause.

Examples:

{[}\textbf{x::y::z} is defined{]} \textbf{x::y::t}

Means that x::y::t value will be ignored unless x::y::z has a value.

{[}\textbf{x::y::z} == ``m\textquotedblright {]} \textbf{x::y::t}

Means that x::y::t will not be used unless x::y::z is defined and
has value ``m\textquotedblright .

\begin{framed}
The rest of this section has been intentionally hidden as it no longer describes pyProCT input script. For 
updated information please visit the repository at \url{https://github.com/victor-gil-sepulveda/pyProCT}.
\end{framed}

